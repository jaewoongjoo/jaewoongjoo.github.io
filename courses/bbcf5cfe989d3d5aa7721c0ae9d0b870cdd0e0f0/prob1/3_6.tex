\PassOptionsToPackage{unicode=true}{hyperref} % options for packages loaded elsewhere
\PassOptionsToPackage{hyphens}{url}
%
\documentclass[]{article}
\usepackage{lmodern}
\usepackage{amssymb,amsmath}
\usepackage{ifxetex,ifluatex}
\usepackage{fixltx2e} % provides \textsubscript
\ifnum 0\ifxetex 1\fi\ifluatex 1\fi=0 % if pdftex
  \usepackage[T1]{fontenc}
  \usepackage[utf8]{inputenc}
  \usepackage{textcomp} % provides euro and other symbols
\else % if luatex or xelatex
  \usepackage{unicode-math}
  \defaultfontfeatures{Ligatures=TeX,Scale=MatchLowercase}
\fi
% use upquote if available, for straight quotes in verbatim environments
\IfFileExists{upquote.sty}{\usepackage{upquote}}{}
% use microtype if available
\IfFileExists{microtype.sty}{%
\usepackage[]{microtype}
\UseMicrotypeSet[protrusion]{basicmath} % disable protrusion for tt fonts
}{}
\IfFileExists{parskip.sty}{%
\usepackage{parskip}
}{% else
\setlength{\parindent}{0pt}
\setlength{\parskip}{6pt plus 2pt minus 1pt}
}
\usepackage{hyperref}
\hypersetup{
            pdftitle={3.6. Convergence of Random Variables},
            pdfborder={0 0 0},
            breaklinks=true}
\urlstyle{same}  % don't use monospace font for urls
\usepackage[margin=1in]{geometry}
\usepackage{graphicx,grffile}
\makeatletter
\def\maxwidth{\ifdim\Gin@nat@width>\linewidth\linewidth\else\Gin@nat@width\fi}
\def\maxheight{\ifdim\Gin@nat@height>\textheight\textheight\else\Gin@nat@height\fi}
\makeatother
% Scale images if necessary, so that they will not overflow the page
% margins by default, and it is still possible to overwrite the defaults
% using explicit options in \includegraphics[width, height, ...]{}
\setkeys{Gin}{width=\maxwidth,height=\maxheight,keepaspectratio}
\setlength{\emergencystretch}{3em}  % prevent overfull lines
\providecommand{\tightlist}{%
  \setlength{\itemsep}{0pt}\setlength{\parskip}{0pt}}
\setcounter{secnumdepth}{0}
% Redefines (sub)paragraphs to behave more like sections
\ifx\paragraph\undefined\else
\let\oldparagraph\paragraph
\renewcommand{\paragraph}[1]{\oldparagraph{#1}\mbox{}}
\fi
\ifx\subparagraph\undefined\else
\let\oldsubparagraph\subparagraph
\renewcommand{\subparagraph}[1]{\oldsubparagraph{#1}\mbox{}}
\fi

% set default figure placement to htbp
\makeatletter
\def\fps@figure{htbp}
\makeatother


\title{3.6. Convergence of Random Variables}
\author{}
\date{\vspace{-2.5em}}

\begin{document}
\maketitle

이 파트부터는 a.e.를 a.s.(Almost Sure Convergence)로 바꿔서 사용한다.

\hypertarget{theoremborel-cantelli-lemma-convergence-half}{%
\paragraph{Theorem(Borel-Cantelli Lemma: convergence
half)}\label{theoremborel-cantelli-lemma-convergence-half}}

If \(\sum_{n=1}^{\infty}P(A_n)<\infty\), then \(P(\limsup_nA_n)=0\).

\begin{itemize}
\item
  확률 공간이라서 (probability는 0부터 1사이 값을 가지므로) summation의
  범위가 1부터로 바뀌었다.
\item
  증명 방법은 똑같다:
  \(\sum_{n=1}^{\infty}P(A_n)<\infty\implies \sum_{k=n}^\infty P(A_k)\rightarrow 0\)
  as \(n\rightarrow\infty\)(꼬리부분은 0이다).
\end{itemize}

\hypertarget{theorem}{%
\paragraph{Theorem}\label{theorem}}

The followings are equivalent:

\begin{enumerate}
\def\labelenumi{\arabic{enumi}.}
\item
  \(X_n\rightarrow X\) a.s.
\item
  \(P(|X_n-X|>\epsilon\mbox{ }\mbox{ }\mbox{ i.o}(n))=0\) for all
  \(\epsilon >0\).
\item
  \(P(|X_n-X|>1/k\mbox{ }\mbox{ }\mbox{ i.o}(n))=0\) for all(integer)
  \(k\ge 1\).
\end{enumerate}

\hypertarget{corollary}{%
\paragraph{Corollary}\label{corollary}}

If \(\sum_{n=1}^\infty P(|X_n-X|>\epsilon)<\infty\) for all
\(\epsilon>0\), then \(X_n\rightarrow X\) a.s.

\hypertarget{theorem-1}{%
\paragraph{Theorem}\label{theorem-1}}

If \(X_n\stackrel{\text{Pr}}\rightarrow X\), then there exists a
subsequence \(\{X_{n_k}:k\ge 1\}\) for which \(X_{n_k}\rightarrow X\)
a.s.

\hypertarget{theorem-2}{%
\paragraph{Theorem}\label{theorem-2}}

A sequence of random variables \(\{X_n:n\ge 1\}\) is Cauchy in
probability if and only if there exists a random variable \(X\) for
which \(X_n\stackrel{\text{Pr}}\rightarrow X\) as
\(n\rightarrow \infty\).

\begin{itemize}
\tightlist
\item
  Completeness for convergence in measure.
\end{itemize}

\hypertarget{theorem-3}{%
\paragraph{Theorem}\label{theorem-3}}

If \(X_n\stackrel{L^p}\rightarrow X\) for some \(0<p\le \infty\), then
\(X_n\stackrel{\text{Pr}}\rightarrow X\).

\begin{itemize}
\tightlist
\item
  \(L^p\) convergence \(\implies\) convergence in measure.
\end{itemize}

\hypertarget{theoremriesz-fisher}{%
\paragraph{Theorem(Riesz-Fisher)}\label{theoremriesz-fisher}}

For \(0<p\le\infty\), the space \(L^p\) is complete: a sequence
\(\{X_n,n\ge 1\}\) of random variables converges in \(p\)th mean(i.e.,
there exists an \(X\in L^p\) s.t. \(||X_n-X||_p\rightarrow 0\)) if and
only if the sequence is Cauchy in \(L^p\)(i.e.,
\(||X_m-X_n||_p\rightarrow 0\) as \(m,n\rightarrow \infty\)).

\begin{itemize}
\tightlist
\item
  Convergenge in \(L^p\) \(\iff\) Cauchy in \(L^p\) (\(L^p\) space is
  complete).
\end{itemize}

\hypertarget{theorem-4}{%
\paragraph{Theorem}\label{theorem-4}}

\(X_n\rightarrow X\) a.s. if and only if
\(\sup_{m\ge n}|X_m-X|\stackrel{\text{Pr}}\rightarrow 0\) as
\(n\rightarrow \infty\).

\begin{itemize}
\tightlist
\item
  Note that \(X_n\rightarrow X\) a.s.\(\iff\)
  \(P(|X_n-X|>\epsilon \mbox{ }\mbox{ }\mbox{ i.o}(n))=0\) for all
  \(\epsilon>0\). Then
\end{itemize}

\[
  \sup_{m\ge n}\left\{|X_m-X|>\epsilon\right\}=  \bigcup_{m= n}^\infty\left\{|X_m-X|>\epsilon\right\}\downarrow \bigcap_{n=1}^\infty \bigcup_{m= n}^\infty\{|X_m-X|>\epsilon  \}=\{|X_n-X|>\epsilon\mbox{ }\mbox{ }\mbox{ i.o}(n)\},\\
\implies  P\left(\sup_{m\ge n}\left\{|X_m-X|>\epsilon\right\}\right)\downarrow P(|X_n-X|>\epsilon\mbox{ }\mbox{ }\mbox{ i.o}(n)),\\
  \implies   \lim_{n\rightarrow\infty} P\left(\sup_{m\ge n}\left\{|X_m-X|>\epsilon\right\}\right)=P(|X_n-X|>\epsilon\mbox{ }\mbox{ }\mbox{ i.o}(n)).
\] Thus, \(X_n\rightarrow X\) a.s. if and only if
\(P\left(\sup_{m\ge n}\left\{|X_m-X|>\epsilon\right\}\right)\rightarrow 0\iff \sup_{m\ge n}|X_n-X|\stackrel{\text{Pr}}\rightarrow 0\).

매우 중요하다.

\hypertarget{corollary-1}{%
\paragraph{Corollary}\label{corollary-1}}

If \(X_n\rightarrow X\) a.s., then
\(X_n\stackrel{\text{Pr}}\rightarrow X\)

\begin{itemize}
\tightlist
\item
  바로 전 Theorem에 의해 \(X_n\rightarrow X\) a.s.일 때 \[
    P(|X_n-X|>\epsilon)\le P(\sup_{m\ge n}\{|X_m-X|>\epsilon\})\rightarrow 0\mbox{ }\mbox{ }\mbox{ as }n\rightarrow \infty.
    \]
\end{itemize}

\hypertarget{theorem-5}{%
\paragraph{Theorem}\label{theorem-5}}

\(\{X_n:n\ge 1\}\) converges a.s. if and only if
\(\sup_{m\ge n}|X_m-X_n|\stackrel{\text{Pr}}\rightarrow 0\) as
\(n\rightarrow \infty\).

\begin{itemize}
\item
  (\(\implies\)): \(X_n\) converges a.s.라 하자.

  그렇다면 for every \(\omega\), \(\{X_n(\omega),n\ge 1\}\)은 Cauchy
  sequence이다, i.e., \(\forall\) \(\omega\),
  \(|X_m(\omega)-X_n(\omega)|\rightarrow 0\) as
  \(m,n\rightarrow \infty\). Thus, \[
  \sup_{m\ge n}|X_m(\omega)-X_n(\omega)|\rightarrow 0 \mbox{ }\mbox{ }\mbox{  a.s. } \mbox{ }\mbox{ } \mbox{ as } n\rightarrow \infty.
  \] Because almost sure convergence implies convergence in probability,
  it follows that \[     
   \sup_{m\ge n}|X_m(\omega)-X_n(\omega)|\stackrel{\text{Pr}}\rightarrow 0 \mbox{ }\mbox{ }\mbox{ }\mbox{ as } n\rightarrow \infty.
   \]
\item
  (\(\Longleftarrow\)):
  \(\sup_{m\ge n}|X_m(\omega)-X_n(\omega)|\stackrel{\text{Pr}}\rightarrow 0\)
  as \(n\rightarrow \infty\)라고 하자. 그렇다면 For any \(m\ge n\), \[
     P(|X_m-X_n|>\epsilon)\le P\left(\sup_{m\ge n}\{|X_m-X_n|>\epsilon\}\right)\\
     \implies \sup_{m\ge n}P(|X_m-X_n|>\epsilon) \le P\left(\sup_{m\ge n}\{|X_m-X_n|>\epsilon\}\right)\rightarrow 0\mbox{ }\mbox{ }\mbox{ }\mbox{ as }n\rightarrow \infty.
     \] Thus, \(\{X_n,n\ge 1\}\) is cauchy in probability \(\iff\)
  \(\exists\) a random variable \(X\) s.t
  \(X_n\stackrel{\text{Pr}}\rightarrow X\) as \(n\rightarrow \infty\).

  Note that \(|X_m-X|\le |X_m-X_n|+|X_n-X|\) for all \(m\ge n\), so that
  \[
     \sup_{m\ge n}\{|X_m-X|\}\le    \sup_{m\ge n}\{|X_m-X_n|+ |X_n-X|\}.
     \] Thus, \[\begin{eqnarray*}
     P\left( \sup_{m\ge n}|X_m-X|>\epsilon\right)&\le&P\left(\sup_{m\ge n}\{|X_m-X_n|+ |X_n-X|>\epsilon\}\right)\\
     &\le&P(\sup_{m\ge n}\{|X_m-X_n|>\epsilon/2\})+P(|X_n-X|>\epsilon/2)\rightarrow 0 \mbox{ }\mbox{ }\mbox{ as }n\rightarrow \infty. 
     \end{eqnarray*}\] Thus, \(X_n\rightarrow X\) a.s.
\end{itemize}

\hypertarget{theorem-6}{%
\paragraph{Theorem}\label{theorem-6}}

\(X_n\stackrel{\text{Pr}}\rightarrow X\) \textbf{if and only if} for
every subsequence \(\{X_{n_k},k\ge 1\}\) there exists a further
subsequence \(\{X_{n_{kj}},j\ge 1\}\) such that
\(X_{n_{kj}}\rightarrow X\) a.s.

\begin{itemize}
\item
  (\(\implies\)): 기존에 했으므로 생략
\item
  (\(\Longleftarrow\)): Note that \[\begin{eqnarray*}
     X_n \stackrel{\text{Pr}}\nrightarrow X&\implies& X_{n_k} \stackrel{\text{Pr}}\nrightarrow X\mbox{ }\mbox{ }\mbox{ for any subsequence }n_k\\
    &\implies& X_{n_{kj}}\nrightarrow X \mbox{ }\mbox{ a.s.}\\
     \iff     X_{n_{kj}}\rightarrow X &\implies&   X_n\stackrel{\text{Pr}}\rightarrow X.\mbox{ (대우명제)}
     \end{eqnarray*}\]
\end{itemize}

\hypertarget{theoremcontinuous-mapping-theorem-for-convergence-in-probability-first-version}{%
\paragraph{Theorem(Continuous Mapping Theorem for convergence in
probability: first
version)}\label{theoremcontinuous-mapping-theorem-for-convergence-in-probability-first-version}}

If \(X_n\stackrel{\text{Pr}}\rightarrow X\) and
\(f:\mathbb{R}\rightarrow\mathbb{R}\) is continuous, then
\(f(X_n)\stackrel{\text{Pr}}\rightarrow f(X)\).

\begin{itemize}
\tightlist
\item
  바로 위의 Theorem에 의해 \[\begin{eqnarray*}
     X_n\stackrel{\text{Pr}}\rightarrow X &\iff& \exists X_{n_{kj}}\rightarrow X \mbox{ a.s. }\mbox{ }\mbox{ }\mbox{ as }j\rightarrow \infty \\
     \implies   X_n\stackrel{\text{Pr}}\rightarrow X &\iff& f(X_{n_{kj}})\rightarrow f(X) \mbox{ a.s. } \mbox{(a.s. convergence is pointwise convergence)}. \\
     \end{eqnarray*}\] Since \(f(X_{n_{kj}})\rightarrow f(X)\),
  \(f(X_n)\stackrel{\text{Pr}}\rightarrow f(X)\) by just the above
  theorem.
\end{itemize}

\hypertarget{lemma}{%
\paragraph{Lemma}\label{lemma}}

Let \(f:\mathbb{R}\rightarrow \mathbb{R}\). Then, the set
\(D_f=\{x\in\mathbb{R} : \mbox{ f is discontinuous at }x\}\) is a Borel
set.

\begin{itemize}
\item
  \(D_f=\{x\in\mathbb{R} : \mbox{ f is discontinuous at }x\}\)도 어떤
  실선 안에서의 interval이므로 쉽게 Borel set이라고 생각할 수 있다
  (\(\phi\)또한 Borel set이다).
\item
  한 discontinuity point, say \(\{a\}\) 또한 Borel set이다. Chapter
  1.1에서 마지막 Borel set에서의 예제를 확인하면 \(\mathcal{I}_0\)에서의
  \(a=b\)일 때와 같다. \(\{a\}\)와 이 set의 여집합의 합은
  \(\mathbb{R}\)이다.
\end{itemize}

\hypertarget{theoremcontinuous-mapping-theorem-for-convergence-in-probability-first-version-1}{%
\paragraph{Theorem(Continuous Mapping Theorem for convergence in
probability: first
version)}\label{theoremcontinuous-mapping-theorem-for-convergence-in-probability-first-version-1}}

Let \(f:\mathbb{R}\rightarrow \mathbb{R}\) be Borel measurable, and let
\(D_f\) be the set of discontinuity points of \(f\).

If \(X_n\stackrel{\text{Pr}}\rightarrow X\) and \(P(X\in D_f)=0\), then
\(X_n\stackrel{\text{Pr}}\rightarrow X\).

\begin{itemize}
\item
  Discontinuity point가 없다는 뜻이 아니라, measure가 0라는 뜻이다.
\item
  즉 interval안에서 point 하나만 discontinuous한 경우(좌극한과 우극한이
  같다) continuous mapping theorem을 사용할 수 있다.
\end{itemize}

\hypertarget{uxc694uxc57d}{%
\paragraph{요약}\label{uxc694uxc57d}}

\begin{enumerate}
\def\labelenumi{\arabic{enumi}.}
\item
  BC-lemma(매우 중요) :
  \(\sum_{n=1}^{\infty}P(A_n)<\infty\implies P(\limsup_nA_n)=0\).
\item
  Riesz-Fisher(중요하다) : \(\{X_n,n\ge 1\}\): Convergenge in \(L^p\)
  \(\iff\) Cauchy in \(L^p\) (\(L^p\) space is complete).
\item
  \(X_n\rightarrow X\) a.s.
  \(\iff \sup_{m\ge n}|X_m-X|\stackrel{\text{Pr}}\rightarrow 0\)
\item
  \(X_n\rightarrow X\) a.s.
  \(\iff \sup_{m\ge n}|X_m-X_n|\stackrel{\text{Pr}}\rightarrow 0\) as
  \(n\rightarrow \infty\)
\item
  \(X_n\stackrel{\text{Pr}}\rightarrow X\) \textbf{if and only if}
  \(\exists\) \(X_{n_{kj}}\rightarrow X\) a.s.(Continuous mapping
  theorem을 위해 필요)
\item
  Continuous Mapping Theorem(매우 중요):
  \(X_n\stackrel{\text{Pr}}\rightarrow X\) and
  \(f:\mathbb{R}\rightarrow\mathbb{R}\) is
  continuous\(\implies f(X_n)\stackrel{\text{Pr}}\rightarrow f(X)\).
\end{enumerate}

\href{../probability1.html}{back}

\end{document}
