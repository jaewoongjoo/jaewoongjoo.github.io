\PassOptionsToPackage{unicode=true}{hyperref} % options for packages loaded elsewhere
\PassOptionsToPackage{hyphens}{url}
%
\documentclass[]{article}
\usepackage{lmodern}
\usepackage{amssymb,amsmath}
\usepackage{ifxetex,ifluatex}
\usepackage{fixltx2e} % provides \textsubscript
\ifnum 0\ifxetex 1\fi\ifluatex 1\fi=0 % if pdftex
  \usepackage[T1]{fontenc}
  \usepackage[utf8]{inputenc}
  \usepackage{textcomp} % provides euro and other symbols
\else % if luatex or xelatex
  \usepackage{unicode-math}
  \defaultfontfeatures{Ligatures=TeX,Scale=MatchLowercase}
\fi
% use upquote if available, for straight quotes in verbatim environments
\IfFileExists{upquote.sty}{\usepackage{upquote}}{}
% use microtype if available
\IfFileExists{microtype.sty}{%
\usepackage[]{microtype}
\UseMicrotypeSet[protrusion]{basicmath} % disable protrusion for tt fonts
}{}
\IfFileExists{parskip.sty}{%
\usepackage{parskip}
}{% else
\setlength{\parindent}{0pt}
\setlength{\parskip}{6pt plus 2pt minus 1pt}
}
\usepackage{hyperref}
\hypersetup{
            pdftitle={L\^{}p-spaces and Inequalities},
            pdfborder={0 0 0},
            breaklinks=true}
\urlstyle{same}  % don't use monospace font for urls
\usepackage[margin=1in]{geometry}
\usepackage{graphicx,grffile}
\makeatletter
\def\maxwidth{\ifdim\Gin@nat@width>\linewidth\linewidth\else\Gin@nat@width\fi}
\def\maxheight{\ifdim\Gin@nat@height>\textheight\textheight\else\Gin@nat@height\fi}
\makeatother
% Scale images if necessary, so that they will not overflow the page
% margins by default, and it is still possible to overwrite the defaults
% using explicit options in \includegraphics[width, height, ...]{}
\setkeys{Gin}{width=\maxwidth,height=\maxheight,keepaspectratio}
\setlength{\emergencystretch}{3em}  % prevent overfull lines
\providecommand{\tightlist}{%
  \setlength{\itemsep}{0pt}\setlength{\parskip}{0pt}}
\setcounter{secnumdepth}{0}
% Redefines (sub)paragraphs to behave more like sections
\ifx\paragraph\undefined\else
\let\oldparagraph\paragraph
\renewcommand{\paragraph}[1]{\oldparagraph{#1}\mbox{}}
\fi
\ifx\subparagraph\undefined\else
\let\oldsubparagraph\subparagraph
\renewcommand{\subparagraph}[1]{\oldsubparagraph{#1}\mbox{}}
\fi

% set default figure placement to htbp
\makeatletter
\def\fps@figure{htbp}
\makeatother


\title{\(L^p\)-spaces and Inequalities}
\author{}
\date{\vspace{-2.5em}}

\begin{document}
\maketitle

\hypertarget{definition}{%
\paragraph{Definition}\label{definition}}

For \(0<p<\infty\), \(L^p(\Omega, \mathcal{F}, \mu)\) is defined to be
the space of all measurable, extended real-valued functions \(f\) for
which \(|f|^p\) is integrable, i.e., \[
L^p(\Omega,\mathcal{F},\mu)= \left\{f:\int |f|^p\mbox{ } d\mu<\infty,\mbox{ }f\mbox{ measurable} \right\}.
\] When the measure space is understood,
\(L^p(\Omega, \mathcal{F}, \mu)\) is often abbreviated to \(L^p\). For
\(1\le p<\infty\), the \textbf{\(p-\)norm} of \(f\in L^p\) is given by
\[
||f||_p=\left[ \int |f|^p\mbox{ } d\mu     \right]^{1/p}.
\] For \(p=\infty\), the infinity norm of \(f\) is defined as \[
||f||_\infty=\inf\left\{\alpha:\mu(\{\omega:|f(\omega)|>\alpha\})=0\right\},
\] where \(\inf\phi:=\infty\). If \(||f||_\infty<\infty\), then \(f\) is
said to be \emph{essentially bounded} and
\(L^\infty=L^\infty(\Omega, \mathcal{F},\mu)\) is defined to be the
space of all essentially bounded functions \(f\).

\begin{itemize}
\item
  \(L^p\) space는 \(|f|^p\) 가 integrable한 measurable function
  \(f\)들의 집합체이다.
\item
  \(p\)-norm은 유한한 \(|f|^p\)의 적분값에 \(1/p\) power를 한
  것이다(Standardized와 비슷한 개념이다).
\item
  Infinity norm은 \(\mu(\{\omega:|f(\omega)|>\alpha\})=0\)에 대한
  \(\alpha\)의 infimum 으로 이는 \(|f|<||f||_\infty\)을 의미한다. 만약
  이 infinity norm이 finite하다면, \(|f|<||f||_\infty<\infty\) a.e.이고
  \(f\)가 essentially bounded 되어 있다고 한다. 또한 만약 이를 만족하는
  \(\alpha\)가 없다면 \(\inf\phi=\infty\)가 되기 때문에, 만약 이
  infinity norm이 존재한다면 이는 finite하고 the space는 essentially
  bounded이다.
\end{itemize}

\hypertarget{lemma}{%
\paragraph{Lemma}\label{lemma}}

If \(0\le a,b<\infty\) and \(0\le \lambda\le 1\), then
\(a^\lambda b^{1-\lambda}\le \lambda a + (1-\lambda)b\), where we take
\(0^0=1\).

\begin{itemize}
\item
  우선 \(-\log\)는 convex function임을 기억하자.
\item
  양변에 \(-\log\)를 씌우면 증명된다.
\end{itemize}

\hypertarget{lemma-1}{%
\paragraph{Lemma}\label{lemma-1}}

If \(c,d\ge 0\), \(1\le p,q<\infty\), and \(\frac{1}{p}+\frac{1}{q}=1\),
then \[
cd\le \frac{c^p}{p}+\frac{d^q}{q}.
\]

\begin{itemize}
\tightlist
\item
  위 lemma에 \(\lambda=\frac{1}{p}\), \(1-\lambda=\frac{1}{q}\),
  \(a=c^p\), \(b=d^q\) 대입.
\end{itemize}

\hypertarget{theoremholders-inequality}{%
\paragraph{Theorem(Holder's
Inequality)}\label{theoremholders-inequality}}

If \(f\in L^p\) and \(g\in L^q\), where \(1\le p,q\le \infty\) and
\(\frac{1}{p}+\frac{1}{q}=1\), then \(fg\in L^1\) and \[
||fg||_1\le ||f||_p||g||_q
\]

\begin{itemize}
\item
  \(p=1\) and \(q=\infty\)일 때: \[
     |fg|=|f||g|\le|f|||g||_{\infty}\implies  ||fg||_1 = \int|fg|\mbox{ }d\mu \le ||g||_q\int|f|\mbox{ }d\mu=||f||_1||g||_\infty.
     \] (Similarly, \(p=\infty\) and \(q=1\)도 성립한다.)
\item
  \(1<p,g<\infty\)일 때: 만약 \(||f||_p=0\)이라면, \(f=0\) a.e.이다.
  때문에 \(fg=0\) a.e.이다(\(||g||_p=0\)일 때도 마찬가지). 때문에
  \(||f||_p,||g||_p>0\)으로 가정한다. 바로 위의 lemma에서
  \(c=|f(\omega)|/||f||_p\) and \(d=|g(\omega)|/||g||_q\)를 대입하면
\end{itemize}

\[\begin{eqnarray*}
cd\le \frac{c^p}{p}+\frac{d^q}{q} &\implies& \frac{|f(\omega)g(\omega)| }{||f||_p ||g||_q}\le \frac{|f(\omega)|^p}{p||f||_p^p} +\frac{|g(\omega)|^q}{q||g||_q^q}\mbox{ }\mbox{ }\mbox{ }\forall \omega\in \Omega\\
&\stackrel{\text{Monotonicity}}\implies& \int \frac{|f(\omega)g(\omega)| }{||f||_p ||g||_q}d\mu\le\int\frac{|f(\omega)|^p}{p||f||_p^p}d\mu +\int\frac{|g(\omega)|^q}{q||g||_q^q}d\mu\\
&\implies& \frac{||fg||_1}{||f||_p ||g||_q}\le \frac{1}{p}+\frac{1}{q} =1 \\
&\implies& ||fg||_1\le ||f||_p ||g||_q.  
\end{eqnarray*}\]

\hypertarget{corollary-cauchy-schwarz-inequality}{%
\paragraph{Corollary (Cauchy-Schwarz
Inequality)}\label{corollary-cauchy-schwarz-inequality}}

If \(f,g\in L^2\), then \(fg\in L^1\) and
\(||fg||_1\le ||f||_2||g||_2\).

\hypertarget{lemma-2}{%
\paragraph{Lemma}\label{lemma-2}}

For any \(a,b\ge 0\) and \(0<p<\infty\), \[
(a+b)^p\le 2^p(a^p+b^p).
\]

\hypertarget{theorem}{%
\paragraph{Theorem}\label{theorem}}

If \(f,g\in L^p\), \(0<p\le\infty\), then \(f+g\in L^p\).

\begin{itemize}
\item
  \(p=\infty\)일 때: \(|f|<||f||_\infty\) a.e. and \(|g|<||g||_\infty\)
  a.e.이다. 때문에 \[
     |f(\omega)+g(\omega)|\stackrel{\Delta}{\le} |f(\omega)|+|g(\omega)|\le ||f||_\infty+||g||_\infty \mbox{ for }\mu\mbox{-almost all }\omega.
     \] 즉 모든 \(\omega\)에 대해서 \(|f+g|\)의 값이 bounded from
  \(||f||_\infty+||g||_\infty\)되어있다. 때문에
  \(||f+g||_\infty\le||f||_\infty+||g||_\infty\)가 존재한다.
\item
  \(0<p<\infty\)일 때: 위의 lemma에 따르면 \[
     |f+g|^p\stackrel{\Delta}{\le}(|f|+|g|)^p\le 2^p(|f|^p+|g|^p)\implies \int|f+g|^pd\mu\le 2^p(\int|f|^pd\mu+\int|g|^pd\mu) <\infty.
     \]
\end{itemize}

\begin{itemize}
\tightlist
\item
  결론: \(L^p\) space는 \textbf{closed under addition}이다.
\end{itemize}

\hypertarget{theoremminkowskis-inequality}{%
\paragraph{Theorem(Minkowski's
Inequality)}\label{theoremminkowskis-inequality}}

If \(f,g\in L^p\), \(1\le p\le \infty\), then \[
||f+g||_p\le ||f||_p+||g||_p.
\]

\begin{itemize}
\tightlist
\item
  \(L^p\) space에서 Triangle inequality라고 생각하자.
\end{itemize}

\hypertarget{proposition}{%
\paragraph{Proposition}\label{proposition}}

If \(c\in\mathbb{R}\) and \(f\in L^p\) for some \(1\le p\le \infty\),
then \(cf\in L^p\) and \[
||cf||_p=|c|||f||_p.
\]

\begin{itemize}
\tightlist
\item
  \(L^p\) space에서 \(p\)-norm안에 있는 상수는 절대값으로 빠져나올 수
  있다.
\end{itemize}

Theorem도 증명도 매우 중요하다.

\hypertarget{theorem-markovs-inequality}{%
\paragraph{Theorem (Markov's
Inequality)}\label{theorem-markovs-inequality}}

For any \(\alpha>0\) \[
\mu(\{\omega:|f(\omega)|\ge \alpha \})\le \frac{1}{\alpha}\int|f|\mbox{ }d\mu.
\]

\begin{itemize}
\item
  증명: \[
     \int |f|\mbox{ }d\mu=\int_{[|f|\ge \alpha]} |f|\mbox{ }d\mu+ \int_{[|f|< \alpha]} |f|\mbox{ }d\mu\ge \int_{[|f|\ge \alpha]} |f|\mbox{ }d\mu\ge \int_{[|f|\ge \alpha]} \alpha\mbox{ }d\mu= \alpha\mu(|f|\ge \alpha)\\
     \implies   \frac{1}{\alpha} \int |f|\mbox{ }d\mu\ge \mu(|f|\ge \alpha).
  \]
\item
  증명으로부터 아래와 같은 finer Inequality를 얻을 수 있다. \[
    \int_{[|f|\ge \alpha]} |f|\mbox{ }d\mu\ge \int_{[|f|\ge \alpha]} \alpha\mbox{ }d\mu= \alpha\mu(|f|\ge \alpha)\\
     \implies  \mu(|f|\ge \alpha) \le  \frac{1}{\alpha} \int_{[|f|\ge \alpha]} |f|\mbox{ }d\mu.
  \]
\end{itemize}

나중에 증명에 사용되니 참고하자.

\begin{itemize}
\tightlist
\item
  우리는 이를 지금까지 elementary course에서
  \(P(|X|\ge \alpha)\le \frac{1}{\alpha}E[|X|]\)로 배웠다. 이를 변형하여
  아래와 같이 얻을 수 있다. \[
    P(|X|\ge \alpha)= P(|X|^k\ge \alpha^k)\le \frac{1}{\alpha^k}E[|X|^k].
     \]
\end{itemize}

\hypertarget{theorem-lyapounovs-inequality}{%
\paragraph{Theorem (Lyapounov's
Inequality)}\label{theorem-lyapounovs-inequality}}

If \(0<\alpha\le\beta\) and \(X\in L^{\beta}\), then \(X\in L^\alpha\)
and
\(\left\{E(|X|^\alpha) \right\}^{1/\alpha}\le \left\{E(|X|^\beta) \right\}^{1/\beta}\).

\begin{itemize}
\item
  \(\alpha=\beta\)일 때 : trivial
\item
  \(0<\alpha<\beta\)일 때 : Take \(p=\beta/\alpha\),
  \(q=1/(1-1/p)=\beta/(\beta-\alpha)\), \(f=|X|^\alpha\), and
  \(g=Y\equiv 1\).

  여기서 \(f=|X|^\alpha \in L^p\)이다. 왜냐하면
  \(\int (|X|^\alpha)^p\mbox{ }d\mu=\int (|X|^\alpha)^\frac{\beta}{\alpha}\mbox{ }d\mu=\int |X|^{\beta}\mbox{ }d\mu<\infty\)이기
  때문이다. \(g\equiv 1 \in L^q\)는 trivial하다. 또한 \(p,q \ge 1\),
  \(1/p+1/q=1\)이다.

  그러므로, Holder's Inequality에 의해 \[
    E(|X|^\alpha)= \int|X|^\alpha d\mu= \int |fg| \mbox{ }d\mu= ||fg||_1 \le ||f||_p||g||_q=({E\left\{|X|^\alpha\right\}^p})^{1/p}= \left\{E(|X|^\beta )\right\}^{\alpha/\beta}\\
    \implies
    \left\{E(|X|^\alpha) \right\}^{1/\alpha}\le \left\{E(|X|^\beta) \right\}^{1/\beta}.
    \]
\end{itemize}

매우 자주 쓰인다.

\hypertarget{theorem-jensens-inequality}{%
\paragraph{Theorem (Jensen's
Inequality)}\label{theorem-jensens-inequality}}

If \(f\) is convex on an interval \(I\subset \mathbb{R}\) containing the
range of the random variable \(X\), and if \(X\) is integrable, then \[
f(E(X))\le E(f(X)).
\]

\begin{itemize}
\tightlist
\item
  2차 함수를 생각하면 쉽다.
\end{itemize}

\href{../probability1.html}{back}

\end{document}
