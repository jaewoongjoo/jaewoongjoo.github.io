\PassOptionsToPackage{unicode=true}{hyperref} % options for packages loaded elsewhere
\PassOptionsToPackage{hyphens}{url}
%
\documentclass[]{article}
\usepackage{lmodern}
\usepackage{amssymb,amsmath}
\usepackage{ifxetex,ifluatex}
\usepackage{fixltx2e} % provides \textsubscript
\ifnum 0\ifxetex 1\fi\ifluatex 1\fi=0 % if pdftex
  \usepackage[T1]{fontenc}
  \usepackage[utf8]{inputenc}
  \usepackage{textcomp} % provides euro and other symbols
\else % if luatex or xelatex
  \usepackage{unicode-math}
  \defaultfontfeatures{Ligatures=TeX,Scale=MatchLowercase}
\fi
% use upquote if available, for straight quotes in verbatim environments
\IfFileExists{upquote.sty}{\usepackage{upquote}}{}
% use microtype if available
\IfFileExists{microtype.sty}{%
\usepackage[]{microtype}
\UseMicrotypeSet[protrusion]{basicmath} % disable protrusion for tt fonts
}{}
\IfFileExists{parskip.sty}{%
\usepackage{parskip}
}{% else
\setlength{\parindent}{0pt}
\setlength{\parskip}{6pt plus 2pt minus 1pt}
}
\usepackage{hyperref}
\hypersetup{
            pdftitle={4.1. Quasi Likelihood Approach},
            pdfborder={0 0 0},
            breaklinks=true}
\urlstyle{same}  % don't use monospace font for urls
\usepackage[margin=1in]{geometry}
\usepackage{graphicx,grffile}
\makeatletter
\def\maxwidth{\ifdim\Gin@nat@width>\linewidth\linewidth\else\Gin@nat@width\fi}
\def\maxheight{\ifdim\Gin@nat@height>\textheight\textheight\else\Gin@nat@height\fi}
\makeatother
% Scale images if necessary, so that they will not overflow the page
% margins by default, and it is still possible to overwrite the defaults
% using explicit options in \includegraphics[width, height, ...]{}
\setkeys{Gin}{width=\maxwidth,height=\maxheight,keepaspectratio}
\setlength{\emergencystretch}{3em}  % prevent overfull lines
\providecommand{\tightlist}{%
  \setlength{\itemsep}{0pt}\setlength{\parskip}{0pt}}
\setcounter{secnumdepth}{0}
% Redefines (sub)paragraphs to behave more like sections
\ifx\paragraph\undefined\else
\let\oldparagraph\paragraph
\renewcommand{\paragraph}[1]{\oldparagraph{#1}\mbox{}}
\fi
\ifx\subparagraph\undefined\else
\let\oldsubparagraph\subparagraph
\renewcommand{\subparagraph}[1]{\oldsubparagraph{#1}\mbox{}}
\fi

% set default figure placement to htbp
\makeatletter
\def\fps@figure{htbp}
\makeatother


\title{4.1. Quasi Likelihood Approach}
\author{}
\date{\vspace{-2.5em}}

\begin{document}
\maketitle

In GLM, \(g(\mu_i)=\sum_j \beta_jx_{ij}\) and likelihood equations are
\[
\sum_i \frac{(y_i-\mu_i)x_{ij}}{\text{Var}(Y_i)}\frac{d\mu_i}{d\eta_i}=0,\mbox{ }\mbox{ } j=0,1,\ldots,p.
\] Let the score function \(\beta\) be \[
S=(S_0(\beta), S_1(\beta),\ldots,S_p(\beta)).
\] * 즉 \(S\)는 log-likelihood \(l(\beta)\)를 각각
\(\beta_0,\ldots,\beta_p\)로 미분한 벡터이다.

Note that ML estimates depend on the distribution of \(Y_i\) only
through \(\mu_i\) and \(\text{Var}(Y_i)=V(\mu_i)\).

~~

\hypertarget{remarkquasi-likelihood-approach}{%
\subparagraph{Remark(Quasi likelihood
approach)}\label{remarkquasi-likelihood-approach}}

\begin{enumerate}
\def\labelenumi{\arabic{enumi}.}
\item
  Use model \(g(\mu_i)=\sum_j \beta_jx_{ij}\) and variance function
  \(V(\mu_i)\) but do not assume distribution for \(Y_i\).
\item
  Use estimating equations \(S(\beta)=0\) even if they do not correspont
  to likelihood equations for distribution in exponential family.
\item
  To allow overdispersion, take \(V(\mu_i)=\phi V^*(\mu_i)\), where
  \(V^*(\mu_i)\) is variance function for common model such as
  \(V^*(\mu_i)=\mu_i\) for count data.
\end{enumerate}

~~

\hypertarget{definition}{%
\subparagraph{Definition}\label{definition}}

A function \(h(Y,\beta)\) is an unbiased \emph{estimating function} if
\[
E[h(Y;\beta)]=0 \mbox{ }\mbox{ for all }\beta.
\] \(S_j(\beta)\) is unbiased estimating function and \(S(\beta)=0\) are
estimating equations.

~~

\href{../glm.html}{back}

\end{document}
